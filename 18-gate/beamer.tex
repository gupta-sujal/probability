\documentclass{beamer}
\usetheme{Madrid}
\usepackage{cite}

%\usepackage[breaklinks=true]{hyperref}
%\usepackage{tkz-euclide} % loads  TikZ and tkz-base
\usepackage{listings}
\usepackage{gvv}
\title{GATE-ST 2023 Q18}
\author{SUJAL GUPTA - EE22BTECH11052}
\institute{EE2102 - IITK (IIT KANDI)}
\date{\today}

\begin{document}
\begin{frame}
\titlepage
\end{frame}
\begin{frame}
\frametitle{Question}
Suppose that $X$ has the probability density function
\begin{align}
f(x)&=
\begin{cases}
\frac{\lambda^{\alpha}}{\Gamma(\alpha)}x^{\alpha - 1} e^{-\lambda x} & \lambda > 0\\
0 & otherwise\\
\end{cases}
\end{align}
where $\alpha > 0$ and $\lambda > 0$. Which one of the following statements is NOT true?
\begin{enumerate}
\item $E(X)$ exists for all $\alpha > 0 $ and $ \lambda > 0$
\item Variance of $X$ exists for all $\alpha > 0$ and $\lambda > 0$
\item $E(\frac{1}{X})$ exists for all $\alpha > 0$ and $\lambda > 0$
\item $E(ln(1+X))$ exists for all $\alpha > 0$ and $\lambda > 0$
\end{enumerate}
\end{frame}

\begin{frame}[allowframebreaks]
\frametitle{Solution}
\begin{enumerate}
\item
{
\begin{align}
E(X)&= \int_{-\infty}^{\infty} xp_X(x)dx\\
&= \int_{0}^{\infty} x\frac{\lambda^{\alpha}}{\Gamma(\alpha)}x^{\alpha - 1} e^{-\lambda x}\\
&= \frac{\lambda^{\alpha}}{\Gamma(\alpha)} \int_{0}^{\infty}x^{\alpha} e^{-\lambda x}\\
\end{align}
since we know that 
\begin{align}
\int_0^\infty x^{\alpha - 1} e^{-\lambda x} dx = \frac{\Gamma(\alpha)}{\lambda^{\alpha}} \qquad \textrm{for } \lambda > 0, \alpha>0
\end{align}
\begin{align}
E(X)&= \frac{\lambda^{\alpha}}{\Gamma(\alpha)}\frac{\Gamma(\alpha+1)}{\lambda^{\alpha+1}}
\end{align}
Using the relation
\begin{align}
\Gamma(x+1) = \Gamma(x) x
\end{align}
\begin{align}
E(X)=\frac{\alpha}{\lambda}
\end{align}
Thus $E(X)$ exists for all $\alpha > 0 $ and $ \lambda > 0$.
}
\item{
\begin{align}
{Var}(X) = {E}(X^2) - {E}(X)^2 
\end{align}
\begin{align}
{E}(X^2) &= \int_{0}^{\infty} x^2 \frac{{\lambda}^{\alpha}}{\Gamma({\alpha})} x^{{\alpha}-1} e^{-{\lambda}x} \, {d}x \\
&=\int_{0}^{\infty} \frac{{\lambda}^{\alpha}}{\Gamma({\alpha})} x^{({\alpha}+2)-1} e^{-{\lambda}x} \, {d}x \\
&=\int_{0}^{\infty} \frac{1}{{\lambda}^2} \frac{{\lambda}^{{\alpha}+2}}{\Gamma({\alpha})} x^{({\alpha}+2)-1} e^{-{\lambda}x} \, {d}x 
\end{align}
\begin{align}
{E}(X^2) = \int_{0}^{\infty} \frac{{\alpha}({\alpha}+1)}{{\lambda}^2}  \frac{{\lambda}^{{\alpha}+2}}{\Gamma({\alpha}+2)} x^{({\alpha}+2)-1} e^{-{\lambda}x}{d}x
\end{align}
using the density of the gamma distribution, we get
\begin{align}
{E}(X^2) &= \frac{{\alpha}({\alpha}+1)}{{\lambda}^2} 
\end{align}
\begin{align}
{Var}(X) &= \frac{{\alpha}^2+{\alpha}}{{\lambda}^2} - {\frac{{\alpha}}{{\lambda}}} ^2 \\
&= \frac{{\alpha}}{{\lambda}^2}
\end{align}
Thus, Variance of $X$ exists for all $\alpha > 0$ and $\lambda > 0$
}
\item {
\begin{align}
E\brak{\frac{1}{X}}&= \int_{0}^{\infty} \frac{1}{x}\frac{\lambda^{\alpha}}{\Gamma(\alpha)}x^{\alpha - 1} e^{-\lambda x}\\
&= \frac{\lambda^{\alpha}}{\Gamma(\alpha)} \int_{0}^{\infty}x^{\alpha-2} e^{-\lambda x}
\end{align}
For this, $\alpha >1$ is a must condition. Hence C is not a correct option.
Hence C is the answer.
}
\item
{
For $E(ln(1+X))$,
\begin{align}
E(ln(1+X))&=E(X)-\frac{E(X^2)}{2}+\frac{E(X^4)}{4}-..
\end{align}
We write the general expression for $E(X^n)$
\begin{align}
E(X^n)&=\frac{\brak{\alpha}\brak{\alpha+1}...\brak{\alpha+n-1}}{{\lambda}^n}
\end{align}
So by applying the ratio test to check the convergence of the sequence
\begin{align}
\lim_{n \to \infty}\biggr \rvert \frac{a_{n+1}}{a_n}\biggr \rvert = L\\
\biggr \rvert \frac{E(X^{n+2})}{E(X^n)}\biggr \rvert&=\frac{\frac{\brak{\alpha}\brak{\alpha+1}...\brak{\alpha+n+1}}{{\lambda}^{n+2}}}{\frac{\brak{\alpha}\brak{\alpha+1}...\brak{\alpha+n-1}}{{\lambda}^n}}\\
&=\frac{\brak{\alpha+n}\brak{\alpha+n+1}}{{\lambda}^2}
%E(X^n)&=\frac{\brak{\alpha}\brak{\alpha+1}...\brak{\alpha+n-1}}{{\lambda}^n}\\
\end{align}
\begin{align}
\lim_{n \to \infty}\biggr \rvert\frac{E(X^{n+2})}{E(X^n)}\biggr \rvert=\infty
\end{align}
Thus $E(ln(1+X))$ generates a divergent function and hence $E(ln(1+X))$ does not exist for all $\alpha > 0$ and $ \lambda > 0$.
}
\end{enumerate}
\end{frame}
\end{document}
