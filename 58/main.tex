\let\negmedspace\undefined
\let\negthickspace\undefined
\documentclass[journal,12pt,onecolumn]{IEEEtran}
\usepackage{cite}
\usepackage{amsmath,amssymb,amsfonts,amsthm}
\usepackage{algorithmic}
\usepackage{graphicx}
\usepackage{textcomp}
\usepackage{xcolor}
\usepackage{txfonts}
\usepackage{listings}
\usepackage{enumitem}
\usepackage{mathtools}
\usepackage{gensymb}
\usepackage[breaklinks=true]{hyperref}
\usepackage{tkz-euclide} % loads  TikZ and tkz-base
\usepackage{listings}
\usepackage{gvv}
\providecommand{\mtx}[1]{\mathbf{#1}}
%
%\usepackage{setspace}
%\usepackage{gensymb}
%\doublespacing
%\singlespacing

%\usepackage{graphicx}
%\usepackage{amssymb}
%\usepackage{relsize}
%\usepackage[cmex10]{amsmath}
%\usepackage{amsthm}
%\interdisplaylinepenalty=2500
%\savesymbol{iint}
%\usepackage{txfonts}
%\restoresymbol{TXF}{iint}
%\usepackage{wasysym}
%\usepackage{amsthm}
%\usepackage{iithtlc}
%\usepackage{mathrsfs}
%\usepackage{txfonts}
%\usepackage{stfloats}
%\usepackage{bm}
%\usepackage{cite}
%\usepackage{cases}
%\usepackage{subfig}
%\usepackage{xtab}
%\usepackage{longtable}
%\usepackage{multirow}
%\usepackage{algorithm}
%\usepackage{algpseudocode}
%\usepackage{enumitem}
%\usepackage{mathtools}
%\usepackage{tikz}
%\usepackage{circuitikz}
%\usepackage{verbatim}
%\usepackage{tfrupee}
%\usepackage{stmaryrd}
%\usetkzobj{all}
%    \usepackage{color}                                            %%
%    \usepackage{array}                                            %%
%    \usepackage{longtable}                                        %%
%    \usepackage{calc}                                             %%
%    \usepackage{multirow}                                         %%
%    \usepackage{hhline}                                           %%
%    \usepackage{ifthen}                                           %%
  %optionally (for landscape tables embedded in another document): %%
%    \usepackage{lscape}     
%\usepackage{multicol}
%\usepackage{chngcntr}
%\usepackage{enumerate}

%\usepackage{wasysym}
%\documentclass[conference]{IEEEtran}
%\IEEEoverridecommandlockouts
% The preceding line is only needed to identify funding in the first footnote. If that is unneeded, please comment it out.

\newtheorem{theorem}{Theorem}[section]
\newtheorem{problem}{Problem}
\newtheorem{proposition}{Proposition}[section]
\newtheorem{lemma}{Lemma}[section]
\newtheorem{corollary}[theorem]{Corollary}
\newtheorem{example}{Example}[section]
\newtheorem{definition}[problem]{Definition}
%\newtheorem{thm}{Theorem}[section] 
%\newtheorem{defn}[thm]{Definition}
%\newtheorem{algorithm}{Algorithm}[section]
%\newtheorem{cor}{Corollary}
\newcommand{\BEQA}{\begin{eqnarray}}
\newcommand{\EEQA}{\end{eqnarray}}
\newcommand{\define}{\stackrel{\triangle}{=}}
\theoremstyle{remark}
\newtheorem{rem}{Remark}

%\bibliographystyle{ieeetr}
\begin{document}
%

\bibliographystyle{IEEEtran}


\vspace{3cm}

\title{
%	\logo{
Solution of GATE-ST 2023 Q58
%	}
}
\author{ SUJAL GUPTA - EE22BTECH11052
}	
%\title{
%	\logo{Matrix Analysis through Octave}{\begin{center}\includegraphics[scale=.24]{tlc}\end{center}}{}{HAMDSP}
%}


% paper title
% can use linebreaks \\ within to get better formatting as desired
%\title{Matrix Analysis through Octave}
%
%
% author names and IEEE memberships
% note positions of commas and nonbreaking spaces ( ~ ) LaTeX will not break
% a structure at a ~ so this keeps an author's name from being broken across
% two lines.
% use \thanks{} to gain access to the first footnote area
% a separate \thanks must be used for each paragraph as LaTeX2e's \thanks
% was not built to handle multiple paragraphs
%

%\author{<-this % stops a space
%\thanks{}}
%}
% note the % following the last \IEEEmembership and also \thanks - 
% these prevent an unwanted space from occurring between the last author name
% and the end of the author line. i.e., if you had this:
% 
% \author{....lastname \thanks{...} \thanks{...} }
%                     ^------------^------------^----Do not want these spaces!
%
% a space would be appended to the last name and could cause every name on that
% line to be shifted left slightly. This is one of those "LaTeX things". For
% instance, "\textbf{A} \textbf{B}" will typeset as "A B" not "AB". To get
% "AB" then you have to do: "\textbf{A}\textbf{B}"
% \thanks is no different in this regard, so shield the last } of each \thanks
% that ends a line with a % and do not let a space in before the next \thanks.
% Spaces after \IEEEmembership other than the last one are OK (and needed) as
% you are supposed to have spaces between the names. For what it is worth,
% this is a minor point as most people would not even notice if the said evil
% space somehow managed to creep in.



% The paper headers
%\markboth{Journal of \LaTeX\ Class Files,~Vol.~6, No.~1, January~2007}%
%{Shell \MakeLowercase{\textit{et al.}}: Bare Demo of IEEEtran.cls for Journals}
% The only time the second header will appear is for the odd numbered pages
% after the title page when using the twoside option.
% 
% *** Note that you probably will NOT want to include the author's ***
% *** name in the headers of peer review papers.                   ***
% You can use \ifCLASSOPTIONpeerreview for conditional compilation here if
% you desire.




% If you want to put a publisher's ID mark on the page you can do it like
% this:
%\IEEEpubid{0000--0000/00\$00.00~\copyright~2007 IEEE}
% Remember, if you use this you must call \IEEEpubidadjcol in the second
% column for its text to clear the IEEEpubid mark.



% make the title area
\maketitle


%\tableofcontents

\bigskip

\renewcommand{\thefigure}{\theenumi}
\renewcommand{\thetable}{\theenumi}

Consider the following regression model
\begin{align}
y_t={\alpha}_0+{\alpha}_1t+{\alpha}_2t^2+\epsilon_{t}, \qquad t = 1,2,…,n
\end{align}
where ${\alpha}_0$ , ${\alpha}_1$ and ${\alpha}_2$ are unknown parameters and $\epsilon_{t}$’s are independent and identically distributed random variables each having $\gauss{\mu}{1}$ distribution with $\mu$ unknown. Then which of the following statements is/are true?
\begin{enumerate}
\item{There exists an unbiased estimator of ${\alpha}_1$}
\item{There exists an unbiased estimator of ${\alpha}_2$}
\item{There exists an unbiased estimator of ${\alpha}_0$}
\item{There exists an unbiased estimator of ${\mu}$}
\end{enumerate}
\solution
Assuming that the model is 
\begin{align}
y_t={\alpha}_0+{\alpha}_1t+{\alpha}_2t^2+\epsilon_{t}
\end{align}
\begin{align}
\begin{bmatrix} 
	y_1  \\
	y_2 \\
	\vdots\\
	y_n  \\
\end{bmatrix}&=\begin{bmatrix} 
	1&A_{11}&A_{12}  \\
	1&A_{21}&A_{22}\\
	\vdots\\
	1&A_{n1}&A_{n2} \\
\end{bmatrix} \begin{bmatrix} 
	{\alpha}_0  \\
	{\alpha}_1 \\
	{\alpha}_2\\
\end{bmatrix}+\begin{bmatrix} 
	\epsilon_{1}  \\
	\epsilon_{2}  \\\vdots\\\epsilon_{n}  \\
\end{bmatrix}
\end{align}
Finding mean of the ${\epsilon}$,
\begin{align}
\vec{y}=A\mtx{\boldsymbol{x}}+\mtx{\epsilon}
\end{align}
\begin{align}
\nonumber E\mathbf{{\epsilon}} = \begin{bmatrix}
E{\epsilon}_1 \\%[5pt]
E{\epsilon}_2 \\%[5pt]
. \\[-10pt]
. \\[-10pt]
. \\[5pt]
E{\epsilon}_n
\end{bmatrix}&=\begin{bmatrix}
\mu \\%[5pt]
\mu \\%[5pt]
. \\[-10pt]
. \\[-10pt]
. \\[5pt]
\mu
\end{bmatrix}
\end{align}
The covariance matrix for ${\epsilon}$ is:
\begin{align}
 \mathbf{C_{\epsilon}}&=\mathbf{E[({\epsilon}-E{\epsilon})({\epsilon}-E{\epsilon})^{T}]} \\
&= E \begin{bmatrix}
({\epsilon}_1-E{\epsilon}_1)^2 & ({\epsilon}_1-E{\epsilon}_1)({\epsilon}_2-E{\epsilon}_2) & ... & ({\epsilon}_1-E{\epsilon}_1)({\epsilon}_n-E{\epsilon}_n) \\
({\epsilon}_2-E{\epsilon}_2)({\epsilon}_1-E{\epsilon}_1) & ({\epsilon}_2-E{\epsilon}_2)^2 & ... & ({\epsilon}_2-E{\epsilon}_2) ({\epsilon}_n-E{\epsilon}_n) \\
. & . & . & .\\
. & . & . & . \\
. & . & . & . \\
({\epsilon}_n-E{\epsilon}_n) ({\epsilon}_1-E{\epsilon}_1) & ({\epsilon}_n-E{\epsilon}_n) ({\epsilon}_2-E{\epsilon}_2) & ... & ({\epsilon}_n-E{\epsilon}_n)^2\\
\end{bmatrix} \\ 
 &= \begin{bmatrix}
\textrm{Var}({\epsilon}_1) & \textrm{Cov}({\epsilon}_1,{\epsilon}_2)& ... & \textrm{Cov}({\epsilon}_1,{\epsilon}_n) \\%[5pt]
\textrm{Cov}({\epsilon}_2,{\epsilon}_1) & \textrm{Var}({\epsilon}_2) & ... & \textrm{Cov}({\epsilon}_2,{\epsilon}_n) \\%[5pt]
. & . & . & .\\
. & . & . & . \\
. & . & . & . \\
\textrm{Cov}({\epsilon}_n,{\epsilon}_1) & \textrm{Cov}({\epsilon}_n {\epsilon}_2) & ... & \textrm{Var}({\epsilon}_n)\\
\end{bmatrix}
\end{align}
Since the ${\epsilon}$'s are independent vectors, 
\begin{align}
\textrm{Cov}({\epsilon}_i,{\epsilon}_j)=0 , \qquad \forall i\neq j
\end{align}
\begin{align}
\textrm{Var}({\epsilon}_i)=1 , \qquad \forall 1\leq i\leq n
\end{align}
Hence, 
\begin{align}
 \mathbf{C_{\epsilon}}&=I_{n\times n}
\end{align}
\begin{align}
E(\vec{y})=E(A\mtx{\boldsymbol{x}}+\mtx{\epsilon})\\
&=AE(\boldsymbol{x})+E({\epsilon})
\end{align}
\begin{align}
Var(\vec{y})=Var(A\mtx{\boldsymbol{x}}+\mtx{\epsilon})\\
&=AVar(\boldsymbol{x})+Var({\epsilon})
&=AVar(\boldsymbol{x})+I
\end{align}
% Here the variable $y_i$ is distributed as \begin{align}\vec{y} \sim \gauss{AVar(\boldsymbol{x})+I}{AE(\boldsymbol{x})+E({\epsilon})}\end{align}
% The maximum likelihood function can be written as:
% \begin{align}
% L\brak{\boldsymbol{x},\mu}&=\frac{1}{{(2\pi)}^{\frac{n}{2}}}e^{\frac{-{(\vec{y}-A\mtx{\boldsymbol{x}})}^T(\vec{y}-A\mtx{\boldsymbol{x}})}{2}}\\
% ln L\brak{\boldsymbol{x},\mu}&=-\frac{n}{2}ln(2\pi)-\frac{1}{2}{(\vec{y}-A\mtx{\boldsymbol{x}})}^T(\vec{y}-A\mtx{\boldsymbol{x}})
% \end{align}
% To find the maximum log likelihood
% \begin{align}
% \frac{\partial lnL\brak{\mtx{\boldsymbol{x}}}}{\partial \mtx{\boldsymbol{x}}}&={A}^TA\boldsymbol{x}-{A}^T\vec{y}\\
% &=0\\
% {A}^TA\mtx{\boldsymbol{x}}&={A}^T\vec{y}\\
% \boldsymbol{\hat{x}} &= (\mathbf{A}^T\mathbf{A})^{-1}\mathbf{A}^T\mathbf{y}
% \end{align}
% Let $B$ be the set of all possible vectors $\alpha$. The object is to find a vector $\alpha$ from $B$ that minimizes the sum of squared deviations of $\epsilon$'s i.e.,
% \begin{align}
% S(\vec{\alpha})&=\sum_{t = 1}^{n} {\epsilon}^2\\
% &={\epsilon}^{T}{\epsilon}\\
% &=\brak{\vec{y}-X\vec{\alpha}}^{T}\brak{\vec{y}-X\vec{\alpha}}
% \end{align}
% Differentiate $S(\alpha)$ wrt $\alpha$
% \begin{align}
% \frac{\partial S\brak{\vec{\alpha}}}{\partial \vec{\alpha}}&=2{X}^TX\alpha-2{X}^T\vec{y}
% \end{align}
% The normal equation is
% \begin{align}
% \frac{\partial S\brak{\vec{\alpha}}}{\partial \vec{\alpha}}&=0\\
% {X}^TX\vec{\alpha}&={X}^T\vec{y}
% \end{align}
\end{document}
